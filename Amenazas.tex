\subsection{Determination of Validity Threats}

En la mayoría de las replicaciones analizadas, no se especifican las amenazas a la validez. Creemos que es conveniente, para cada uno de los cambios, analizar el tipo de amenaza afectada. 
En \cite{gomez2014understanding} se analiza como afectan, en las amenazas a la validez, los cambios de la configuración experimental en relación a la línea base:

\begin{itemize}
    \item Cambios en el \textit{protocolo}: afectan a la \textit{validez interna}. Al cambiar el protocolo, si se confirman los resultados, se verifica que lo observados es cierto y no consecuencia de la configuración del experimento.
    
    \item Cambios en la \textit{operacionalización}: afectan a la \textit{validez del constructo}. Si se cambia una variable dependiente por otra que refleja mejor lo que se está midiendo, aumenta la \textit{validez del constructo}. En general, si se suprime/añade alguna variable dependiente o independiente, disminuye/aumenta la \textit{validez del constructo}. Al aumentar el número de variables el procedimiento puede volverse más tedioso y disminuir la \textit{validez de conclusión} \cite{wohlin:experimentation}.

 \item Al suprimir, en un experimento de inspección, alguno de los documentos utilizados diminuye \textit{la validez del constructo}. %\cite{wohlin:experimentation}pag. 109. Podemos decir que al suprimir documentos/programas disminuye Validez de constructo.
    
     \item Cambios en la \textit{población}: aumentan la \textit{validez externa}. Permite conocer para que tipo de población se mantienen los resultados.
     
      \item Cambios en los \textit{experimentadores}: permiten saber si los resultados son independientes de los experimentadores. Puede ser que los experimentadores, al intentar demostrar una hipótesis, influyan en los resultados; al cambiar los experimentadores se evitaría esta amenaza y por tanto aumenta la \textit{validez de conclusión} \cite{wohlin:experimentation}.
    
    \item Sin cambios, aumenta la muestra: aumenta la \textit{validez de conclusión}. Si la \textit{población} es del mismo tipo, al aumentar la muestra puede disminuir la \textit{validez externa} \cite{wohlin:experimentation}. 
\end{itemize}
