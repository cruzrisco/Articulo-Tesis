% ----------------------------
% File    : T-DSR.tex
% Content : Tabla DRS
% Date    : 1/12/2018
% Version : 1.0
% Authors : M.Cruz
% ----------------------------
\begin{table*}
  \caption{Activities of the \DSR methodology and corresponding sections}
  \label{tab:fases-DSR}
  \centering
  \scriptsize
  \begin{tabularx}{0.98\textwidth}{
    >{\hsize=0.4\hsize}X
    >{\hsize=0.9\hsize}X
    >{\hsize=0.23\hsize}X}
 
    \toprule
    Actividad  &
    Descripción   &
    Sección \\
    \midrule
    
    \emph{Problem identification and motivation}  &
    Need to address an accurate definition of changes for the reporting of replications of controlled experiments & 
    \ref{sec:intro}  \\ 
    
    \emph{Solution objectives definition}  & 
    Facilitate documentation of changes so that they are clearly specified. Literature is reviewed to: i) analyse how changes are being reported; ii) identify the information involved; and, iii) know other proposals &
    \ref{sec:SLR} \\
    
    \emph{Design and Development}  & 
    A meta--model is proposed. A possible implementation of the meta--model is the template completed with L-patterns & 
    \ref{sec:metamodelo}, \ref{sec:plantilla} \\
    
    \emph{Demonstration}  &
    The first version of the template is instantiated by defining the changes of a family of experiments belonging to the \gls{SE} area.  &
    \ref{sec:Mind} \\
    
    \emph{Evaluation}  &
    The artefact, in our case the template, is evaluated in a multiple case study encompassing various areas of knowledge and types of experiments. CÆSAR tool is used as a proof of concept. & 
    \ref{sec:Case},  \ref{sec:SoftEng-Case}, \ref{sec:Science-Case}, \ref{sec:Automatic-Case} \\
    
    \emph{Communication} & 
    The results of the research are published &  \\ 
    
    \bottomrule
  	\end{tabularx}   
\end{table*}