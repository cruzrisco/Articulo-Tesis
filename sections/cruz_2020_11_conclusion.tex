% --------------------------------------------------
% File    : cruz_computing_2020_11_conclusion.tex
% Content : Conclusions and Future work section of the article for Computing
% Date    : 1/1/2019
% Version : 2.0
% Authors : M.Cruz
% --------------------------------------------------

%---------------------------------------
\section{Conclusions and Future work}
\label{sec:conclusions}
%---------------------------------------

A meta--model, visually represented by a template completed with L-patterns, has been presented to facilitate the specification of replication changes in controlled experiments. 
The proposal improves the design of the replications as it allows for in-depth design and analysis of the implications.

The template has been evaluated by specifying the changes in the replications that constitute a multiple--case study. To facilitate the filling in of the template, the CÆSAR tool, developed by the authors of this document, has been used. Three case studies have been included in the multiple--case study (SoftEng-Case, Science-Case and Automatic-Case) encompassing several areas of knowledge and types of experiments.

By means of the multiple--case study, \emph{expressiveness, precision, usability} and \emph{traceability} of the proposed template have been analysed:

Limitations were detected in the \emph{expressiveness} of the template and it was consequently improved. %
After the adjustments, the current version of the template has been able to fully \emph{express} the replications of the multiple--case study and, in particular, all its changes. %
In addition, the \emph{expressiveness} of the template  with respect to other proposals was analysed.

It has been confirmed that the specification of changes using the template and L-patterns is more \emph{precise} than in natural language, relevant information is reflected (reducing tacit knowledge), and there are no \emph{under--specified changes}.

The problems of \emph{usability} are related to the different terminology used. In \emph{Automatic experiments} and, in general, in the \gls{SE} area it is common practice to analyse the \emph{validity threats}. However, in the science area, the expressions \emph{validity threats} and \emph{dimension affected} are not used. 
The CÆSAR tool improves the \emph{usability} of the template by presenting the different options available as well as providing an overview of the family of experiments.


The \emph{traceability} or historical evolution of the experiment is facilitated since the template specifying its replications becomes part of the \emph{laboratory package}, making the experiments more reproducible.

%The first \emph{SoftEng-Case} was focused in \gls{SE} knowledge area. The problems encountered in specifying both the basic aspects of replication and its changes have led the template evolve from its initial version. New fields have been added and the wording or phrases that are proposed as parameters in some of the defined L-patterns are been improved. The underlying meta--model has also been modified. 

%In the second \emph{Science-Case} the template has been validated in science knowledge area. Despite the effort required to complete the template, it is offset by the benefit of recording each change and its cause, including changes imposed by the medium (e.g. use of available reactive). When the template is not used, in many cases, the changes are not recorded and are only known to the experimenter.

%The third \emph{Automatic-Case} has allowed to demonstrate the usefulness of the template specifying \emph{Automatic experiments}. All changes could be fully specified including the \emph{justification for the change} and \emph{validity threats}.

As future work, our aim is: \emph{i)} extend the proposal for the summary specification of complete replication and not just of the changes;
\emph{ii)} evolving to a stable version of the CÆSAR tool beyond a proof of concept, as well as a virtual assistant proposing the effect on the \emph{validity} of the \emph{dimension affected} by the change;
\emph{iii)} encourage and disseminate the use of the template in our community to perform an external validation of the same; and
\emph{iv)} to typify the elements of the new \emph{Context dimension} identified in this work.

%Further, our intention is to extend the experimental information repository EXEMPLAR \cite{ParejoExemplar2014} with aspects of replications based on meta--model and template proposed and reviewed in this paper. Our objective is to achieve a global vision of the family of experiments, to be able to generate the experimental setting of a replication in an agile way from the original experiment and, in short, to offer more guarantees on the validity of the description of a replication. %
%Families of experiments are on the rise in Software Engineering \cite{santos2018analyzing} and are defined according to Basili \etal \cite{basili1999building} as a group of experiments that pursue the same goal to extract mature conclusions.

