% ------------------------------------------------------
% File    : cruz_2020_00_abstract.tex
% Content : Abstract section of the article 
% Date    : 16/07/2018
% Version : 1.0
% Authors : M. Cruz
% ------------------------------------------------------
% ------------------------------------------------------
% Last update: 22/11/2020
% ------------------------------------------------------

\begin{abstract}
\emph{Context}: Replication of empirical studies in Software Engineering is necessary to consolidate acquired knowledge and generalize results from related studies. To reap the benefits of replications, information must be published in a way that allows a better understanding of the differences between the replication and the baseline study. %
%
\emph{Objective}: 
The goal of the present work is to facilitate documentation of replication changes and, therefore, the replicability of experiments in Empirical Software Engineering. %
%
\emph{Method}: First, a meta--model was presented to formalize related concepts and a template with linguistic patterns was designed to visually display the meta--model information. 
Based on the proposed template, the CÆSAR tool was developed to facilitate the definition of changes and provide an overview of the family of experiments.
The template was evaluated by specifying changes in replications of several families of experiments in a multiple--case study encompassing various areas of knowledge and types of experiments. The application of the template in other areas has made possible to analyze the difficulties encountered and compare the terminology and concepts used.
\emph{Results}: 
By means of the multiple--case study, the expressiveness, precision, usability and traceability of the proposed template have been analyzed. 
The limitations encountered have allowed the template to evolve from its initial version and, in spite of the different terminology and concepts used in the area of sciences, the improved template has allowed to define changes systematically and homogeneously. 

\keywords{%
Replication \and 
Meta--model \and 
Template 		\and 
Linguistic patterns \and
Empirical study     \and 
Family of experiments}

\end{abstract}