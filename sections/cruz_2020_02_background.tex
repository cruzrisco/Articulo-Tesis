% ------------------------------------------------------
% File    : cruz_computing_2020_02_background.tex
% Content : Introduction (y DSR) section of the article for Computing
% Date    : 19/11/2020
% Authors : M.Cruz
% ----------------------------------------------

\section{Background}
\label{sec:background}

Before presenting our proposal, it is first necessary to point out the importance of replications and then to introduce some concepts used to specify the replications.

To justify the importance of replications, below are some organisations that have published or taken an interest in the subject of replications and its related aspects.

In \cite{acm}, the Association for Computing Machinery (ACM) recommend that three separate badges related to artifact review be associated with research articles and classifies them as: \emph{i) artifacts evaluated}, \emph{ii) artifacts available} and \emph{iii) results validated}. In these last ones, the same results have been obtained in a subsequent study conducted by people other than the authors. Two levels are distinguished: \emph{results reproduced} and \emph{results replicated}, depending on whether or not the artifacts provided by the author are used.

ACM stresses that an experimental result is not fully established unless it can be independently \emph{reproduced}, noting furthermore, the benefits obtained by making research artifacts available to the public to facilitate replications that verify the robustness of the original results.

Similarly, at ICSE (International Conference
on Software Engineering) it is noted that reviewing the artifacts of accepted papers increases the likelihood that the results can be  \emph{replicated} and \emph{reproduced} by other researchers. 
ICSE invites the authors of accepted contributions to present the associated artefacts for evaluation and to receive one of the ACM badges.  Authors can also make short presentations at the ROSE (Recognizing and Rewarding Open Science in Software Engineering) festival. ROSE festivals are held at major conferences in the area.

Finally, to highlight the \emph{Open Science} initiative to make research data public, thereby increasing the \emph{transparency} and \emph{reproducibility} of the studies \cite{fernandez2019open}.
\emph{Open Science} is based on: \emph{i)} open access to the articles, \emph{ii)} open data and \emph{iii)} open source software.

% --- Dimensiones ---

Concerning the specification of replications, it is necessary to go deeper into the nature of the changes involved. 
A change can affect an element of the experimental configuration. It is necessary to identify the elements that make up the experimental configuration and which of these elements can be changed and still be considered as the same experiment.
In \cite{gomez2014understanding}, elements of the experimental configuration are classified into the following groups called dimensions: \textit{Operationalization}, \textit{Population}, \textit{Protocol} and \textit{Stakeholder}. 
There is also a fifth \emph{Context} dimension identified by us.

In a controlled experiment, there are the constructs of \emph{cause} and \emph{effect} (or also called independent and dependent variables).
The \textit{Operationalization} dimension includes the translation of the \emph{cause} and \emph{effect} constructs into their concrete manifestations.
The \emph{treatment} is the \textit{Operationalization} of the \emph{cause}.
\emph{Metrics} and \emph{measurement} procedures are the \emph{operationalization} of the \emph{effect}.

The \emph{Population} dimension includes the \emph{experimental subjects} that act on the experimental \emph{objects}. 
The treatments are applied to the combination of \emph{objects} and \emph{experimental subjects}.

The \emph{Protocol} dimension is the configuration of \emph{experimental design}, \emph{experimental material}, \emph{guides}, \emph{iv) measuring instruments} and \emph{data analysis techniques} to observe a particular effect.

The \emph{Stakeholder} dimension includes the researchers participating in the experiment in their different \emph{roles: designer, analyst, trainer, monitor} and \emph{ measurer}.

In this article, the \emph{Context} dimension related to the environment in which the experiment is carried out is proposed. Changes in the context affect the results.
For example, carrying out the replication at a different time or day may affect the results or carry it out at exam time instead of at the beginning of the course. 
