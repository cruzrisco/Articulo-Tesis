% --------------------------------------------------
% File    : cruz_computing_2020_10_trabajos.tex
% Content : Related work section of the article for Computing
% Date    : 1/12/2018
% Authors : M.Cruz
% --------------------------------------------------

%---------------------------
\section{Related work}
\label{sec:trabajos}
%----------------------------

This section reviews some templates and patterns defined or used in the context of SE.

First, the Goal-Question-Metric (GQM) template proposed in 1994 by Basili et al. \cite{Basili1994} widely used and recommended by Wohlin \cite{wohlin:experimentation} for the definition of the experiment's goal. 

The idea of using templates and patterns to define changes in the replication of experiments is based on the templates proposed, in 1999, by Durán et al. \cite{duran1999requirements} and used in \cite{duran2002supporting} in the implementation of REM, an experimental requirements management tool based on XML. 
The empirical assessment, conducted in \cite{bernardez2004controlled}, is based on empirical data collected from requirement documents developed by students using the REM tool. As a result, the underlying meta--model is improved, and important feedback is provided.

The templates and L-patterns have also been successfully applied in several areas within SE. In 2012-2016, Del Rio et al. \cite{del2012defining,del2016using} have used them for the definition of Process Performance Indicators (PPIs).

In the DSR methodology, Wieringa  \cite{38631e0608b54d4299d5707f3a78debf}, in 2014, similarly to our work, defines a template with patterns for the specification, in that case, of a problem that will lead to future research. In addition, the DSR methodology proposes a series of phases to guide the research and are those that have been followed in the present work (see Table \ref{tab:fases-DSR}).

Finally, Segura et al.  \cite{segura2017template}, in 2017, propose a template inspired by the GQM template for the definition of metamorphic relationships. The format is textual rather than tabular to facilitate integration into research work.
