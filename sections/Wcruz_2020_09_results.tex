% ----------------------------------------
% File    : cruz_computing_2020_09_results.tex
% Content : Case Study results
% Date    : 1/12/2018
% Authors : M.Cruz
% ----------------------------------------

% -----------------------------------
% Movido a sección aparte
% Last update: 16/11/2020 (MCruz)
% Actualización a partir de la tesis
% -----------------------------------  

%--------------------------------
\section{Multiple--case studies results}
\label{sec:CS-Results} 
%--------------------------------
\rojo {LO he movido}
In this section we are going to structure the results to answer the \gls{RQs} proposed in section \ref{sec:CS-design}.

\begin{itemize}
%----------------------
%  RQ1: Expressiveness
%---------------------- 
    \item[•] RQ1: \emph{\Expressiveness}. It is analyzed from two aspects:
	
	\begin{enumerate}
        \item  When defining the replications of the first \SoftEng-Case with the initial version of the template, limitations were detected in the \emph{\expressiveness} of the template and it was consequently improved.
	
        Limitations of the template are related to the inability to express: 
    
        \begin{itemize}
	        \item Site and date of the baseline experiment and its replication.
	        \item Description of the experiment.
	        \item Acronym for baseline experiment and replication.
	        \item Replications on a previous replication.
	        \item \emph{Validity threats} affecting some changes.
	        \item Changes affecting context variables.
	        \item The reason for the change is unknown.
	        \item The name of the variable affected by the change is unknown.
        \end{itemize}

    Sections \ref{sec:SE-basic} and \ref{sec:SE-changes} detail the description of these problems along with the adjustment made on the template and/or meta--model to express what the experimenter needs reflected. 
    After the adjustments, the current version of the template has allowed to express the replications of the families belonging to the three case studies and specifically all their changes. 


    \item In addition, it is of interest to analyse the \emph{\expressiveness} of the template with respect to other proposals followed by the authors of replications. Table \ref{tab:CS-Guides} compares our proposed template with the guidelines followed in the studies analysed in the revision presented in section \ref{sec:SLR}. \\
    
\input{figures/T-MultiCS-Guides.tex}  

    First, guidelines have been classified according to whether they are intended to report on experiments (EX), case studies (CS) or specifically replications (RE).
    Its \emph{\expressiveness} is then analysed by addressing aspects such as: 

    
    \begin{itemize}
        
	    \item \emph{Experiment motivation}. In all guidelines, it is recommended to report the \emph{motivation of the experiment}. 
	
	    \item \emph{Changes to the original experiment and its reason}. Only in Carver's guidelines \cite{carver2010towards} is it recommended to report the \emph{changes to the original experiment} along with your motivation. Similarly, Juristo and Moreno \cite{juristo2013basics} when referring to previous experiments, recommend to indicate \emph{the altered characteristics}.
	
	    \item \emph{Validity threats}. Another of the issues that should be addressed in the experiment report are the \emph{threats to validity}. It's mentioned in all the guidelines except Carver's.
	    
	    \item \emph{Use of L-pattern}. None of the guidelines analysed use templates. In our proposal, the use of L-patters facilitates the definition of change and therefore increases \emph{\expressiveness}.

    \end{itemize}
    
      
    
    In addition to these guidelines, Gómez \etal \cite{gomez2014understanding} address in depth the characteristics of the changes in replications but does not provide a guideline for the report so it is not included in the table. 
    
    The proposed template include both the \emph{purpose} for carrying out the replication and the definition of the change together with the specific \emph{motivation}.
    In addition, the definition of change is completed with the identification of the \emph{experimental dimension} affected and the influence of the change on the \emph{validity} of the experiment is analysed.
    
    The full specification of the changes facilitates new replications, as corroborated by comments from experimenters in other areas involved in the multiple--case studies.
    
    \end{enumerate}

%----------------------
%  RQ2: Precision
%---------------------- 
    \item[•] RQ2: \emph{\Precision}. Specification of changes by means of the template and L--patterns is more precise than in natural language, as it allows the lack of relevant information to be detected.
    This lack of information also includes \emph{tacit knowledge} described as knowledge that the researcher does not make explicit in the experiment report \cite{shull2002replicating}.
    
    In \SoftEng-Case, we realised that there were \emph{under-specified changes} because the \emph{reason for the change} or the \emph{name of the variable} affected by the change were not specified in the report of the replication.

    Likewise, the \precision has also allowed us to realize that there are fields in the template that are difficult to complete , as confirmed by applying the template in \Science-Case, where the \emph{threats to validity} and \emph{dimension affected} by the change have only been completed in one of the five families of experiments. \\ % -- helped by us--. \newline
    
%--------------------------
%  RQ3: Understandability
%--------------------------
    \item[•] RQ3: \emph{\Understandability}. 
    In order to verify the \understandability of the template it is necessary that it is used by external experimenters.
    To achieve this, the \Science-case and \Automatic-case were designed with the main objective that the experimenters who have carried out the replications fill in the template. The commentaries received from the experimenters are included in sections \ref{sec:SC-discussion} and \ref{sec:AUT-discussion}.
    
    Table \ref{tab:CS-Compara} compares the use of the template considering all families of experiments analysed in this multiple--case study
    
\input{figures/T-MultiCS-Compara.tex}
    
    For each family of experiments, the \emph{number of replications}, \emph{number of changes} and whether the \emph{reason for each change}, the \emph{modified dimension} and the \emph{validity threat} have been completed are shown.
    
    It should be noted that the 95 changes have been fully specified, including the \emph{reason for the change}; however, fields such as \emph{dimension affected} and  \emph{threats to validity}, have not been filled in families of experiment belonging to \emph{\Science-Case} as they are not explicitly defined in these disciplines.

    In the Family \emph{Soil}, we have collaborated with the experimenter to ensure that \emph{validity threats} and \emph{affected dimensions} are identified.

    The \emph{reason for the change} has not been fulfilled in some \emph{Req} family changes because it was not specified in the replication report.

    Differences in the use of the template may be due to:

    \begin{itemize}
	    \item \SoftEng-Case. The template has been proposed in the \gls{SE} area and was filled in by us from the replication report, so logically there have been no problems of \understandability.   
	
	    \item \Science-Case. Experimenters have filled in the template and are unfamiliar with some of the terminology used, so it has been laborious or impossible to fill in fields such as \emph{validity threats} and \emph{dimension affected}. 
	
    	On the other hand, the rest of the template has not presented problems of \understandability, mainly because of the help of the L--patterns that facilitate the drafting of the changes.
	
	    \item \Automatic-Case. The families in which the template has been instantiated belongs to the \gls{SE} area so the terminology is known and the template is well suited to this type of experiments.  \\

    \end{itemize}	
    
%--------------------------
%  RQ4: Traceability
%--------------------------
    \item[•] RQ4: \emph{\Traceability}. 
    The specification, using the template, of the replications that constitute the family of experiments, allows us to know the evolution of the experiment. The use of an acronym to identify replication simplifies the trace between elements.
    
    \Traceability facilitates new replicas since by recording both the \emph{reason for the replication} and the \emph{reason for each change}, the changes already made are known and new changes can be designed and proposed. When changes need to be made to adapt the experiment to a new environment, it may be useful to analyse the changes already made in environments with similar conditions.
     
    On the other hand, the template should be part of the \emph{laboratory package} reflecting the evolution of the experiment and establishing the trace between successive replications of the same family of experiments. \\

\end{itemize}


